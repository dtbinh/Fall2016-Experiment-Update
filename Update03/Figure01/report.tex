\documentclass{article}
\usepackage{graphicx} 
\begin{document}
\section*{A brief report of experiment results last week}
\vspace{2cm}
\subsection*{a. Tracking Results}
Seven of eleven attempts are not fail last weekend, and five of them are worth to show. 
According to the tracking behavior about the robot, experiments no.1(blue),5(yellow),6(purple) in figure \ref{figTrackGood} is reasonable. 
In those results agent may get lost temporarily, and all the agents will finally move towards the source.
\begin{figure}[htbp]
\centering
\includegraphics[width=12cm]{TrackingresultGood} 
\caption{Tracking Results of Khepera Robot}\label{figTrackGood}
\end{figure}
\\In figure \ref{figTrackGood}, and in figure \ref{figTrackPoor}, the tails correspond to the tracking trajectories of robots, 
and the short line in their heads are shown the final orientation of those Khepera robots. 
Symbol '+' in the center of every four agents is the geographic center of them, and another symbol '*' corresponds to the estimation direction for those groups. \\
Experiment no.2(cyan) and experiment no.7(gold), in figure \ref{figTrackPoor}, are failed to find the source with gradient results. 
The boundary in the corner of $(2,2)$ may possible influence the results.
\begin{figure}[htbp]
\centering
\includegraphics[width=12cm]{TrackingresultPoor} 
\caption{Tracking Results of Khepera Robot}\label{figTrackPoor}
\end{figure}
\subsection*{b. Estimation and Error}
Comparing the results from experiment no.1,5,6 or the results from experiment no.2,7, there are no significant differences of estimation 
shown in figure \ref{figEstGood} and figure \ref{figEstPoor}. 
\begin{figure}[htbp]
\centering
\includegraphics[width=12cm]{EstimationresultGood} 
\caption{Estimation Results of Khepera Robot}\label{figEstGood}
\end{figure}
\begin{figure}[htbp]
\centering
\includegraphics[width=12cm]{EstimationresultPoor} 
\caption{Estimation Results of Khepera Robot}\label{figEstPoor}
\end{figure}
\\
It seems that, if the group of robots fail to reaching the source, the estimation results may become more diverse than they success. 
However, we cannot say that the estimations are directly or explicitly influenced by the gradient. 
One possible reason for this phenomenon is that the esitmation and the calculation of gradient are both influenced by the boundary of our experimental field. \\
Another interesting phenomenon shown in figure \ref{figEstPoor} is that 
the blue line, which represents the first experiment, jumps to another value after a stable result. 
Considering the result of another two experiment shown in this figure, we can say that the diffusion coefficient jumps from a true field value, 
and there are no signs show that this coefficient will jump back to reach consensus. 
I think we may need more discussions on this estimation coefficient. \\
I have recorded time series in those experiment. 
However, as shown in figure \ref{figEstGood} and figure \ref{figEstPoor}, it is hard to define a consensus speed for those coefficient, especially for 
experiment 1, because we cannot garantee that the estimation value will no longer jump again. 
And for other experimental attempts, the consensus time of the estimation seems random or in some case they consensus too fast.
\\
For concentration errors, all the results seems reasonable. 
In figure \ref{figErrGood} and figure \ref{figErrPoor}, a possible sensor failure may cause the sudden change of the estimation error. 
These kinds of error may not influence the tracking and the estimation our agents.
\begin{figure}[htbp]
\centering
\includegraphics[width=12cm]{ErrorresultGood} 
\caption{Concentration Estimation Error of Khepera Robot}\label{figErrGood}
\end{figure}
\begin{figure}[htbp]
\centering
\includegraphics[width=12cm]{ErrorresultPoor} 
\caption{Concentration Estimation Error of Khepera Robot}\label{figErrPoor}
\end{figure}
\end{document}